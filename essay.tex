\title{The ethics of toxicity in the online community}

\author{1601002}

\date{\today}

\documentclass[12pt]{article}

\begin{document}
\maketitle
Should video game developers take action against toxic players?

\begin{abstract}
Video game developers are known to hand out punishments to players that are misbehaving in their game, be it online or offline. Although the developers do have an obligation to keep the game enjoyable for as many people as possible, is this ethically feasible? Compared to a very primitive example of a moral decision such as The Trolley Problem, which in this case is the online community, either outcome results in at least 1 individual being hurt. Objectively speaking, the Trolley Problem has no right or wrong, however griefing and flaming is seen as inherently wrong in our culture. This paper looks to explore the connections between the moral choices in video games that extend from singleplayer offline gameplay to online peer to peer communication.
\end{abstract}


\section{Introduction}
Video games have quickly become more relevant in the modern day world than books as a survey suggests that in Australia, 2011, 92 percent of homes had access to a video gaming console\cite{Martin:2011:PGE:2071536.2071566}. This, combined with the online disinhibition effect which describes people's personalities and attitudes changing after they have logged into the online world\cite{Zim}, creates a platform for misconduct due to the actions of the user not baring repercussions. This means that games which combine the MMORPG aspect with moral decision making such as \textit{Star Wars The Old Republic} could result in some players experimenting with moral and ethical decisions that shape the personality they wish to attain virtually. This paper aims to take the moral decision making of video games and see whether the solutions players come to in game could potentially sway their real personality.

\section{Transferable morals}
In most games that involve the ethical mechanic, comitting violent acts rewards the player with "evil" or "good" points, these points can then open up more gameplay options depending on the stance the player has taken. A survey done on students in a classroom posed the scenario: \textit{"A student has been placed in a group which is culturally very different from the one he or she belongs to (this could be gender, race, sexual orientation or ethinicity), do you:"}, the survey then offers 5 responses each classed as a different moral standpoint \cite{7942917}. The results found that 35 percent took the pragmatic view, 31 percent the utilitarianist view and 27 percent the hedonist view. Interestingly, option 5 was believed to be the most popular prior to the survey as it involved listening to the problem as described by the student and helping the student and the group work together (this could be considered the 'good' choice in a game), while the hedonist choice to tell the student \textit{"That's life"} could be considered the 'evil' choice. The validity of this survey is called into question however, as in a follow-up interview the results from the same participants changed.In the early 1990s a company called Lockheed-Martin developed a board game designed to test the ethics of their workers but found that genuine cases from the company's history were more effective for the ethics game than hypothetical scenarios \cite{4408567}. This shows that although a game based around ethics can be used to teach and transfer ethically sound morals, the game does still require legitimate examples of scenarios.

\section{Subjective morals}
However, what is ethically 'good' can vary from culture to culture \cite{7426669}. Based on Zimbardo's model for disinhibited behavior online \cite{Zim}, which outlines the inverse ratio of anonymity to identity (i.e the more anonymous a person becomes the more they lose their real identity), the work done to establish a person's morals based on an ethics based game could be called into question as the answers given by the person are not true to their personality in reality. Furthermore, video games have been shown to relieve people of stress \cite{7539750}, however one of the ways this is done is by player's venting their negative emotions which they have picked up throughout the day in the real world \cite{6701985}. If the player venting their emotions undertakes this task on an MMORPG then it's possible that the target of their anger is another player, which is considered bullying and/or griefing \cite{coyne2009griefing}.

Coyne believed that players show more aggressive behaviour towards others online as there is a lack of face to face interaction, therefore the player does not see other players as real. In 2015 Kotaku \cite{hernandez_2015} published a report stating a popular YouTuber \textit{VideoGameDunkey} was banned from a MOBA game called \textit{League Of Legends} \cite{leagueoflegends} for 'flaming'. The YouTuber (hence, Dunkey) stated flaming others was "the only fun part of the game" and that the developers \textit{Riot Games} wanted people to behave like robots.

This scenaro does raise a valid point with the subjectivity of morals from different people's backgrounds, even if a player follows all the rules set by the developer, they can still be considered Toxic \cite{Kwak:2014:UTB:2567948.2580066} due to just exhibiting certain emotions. On the other hand, there was a study \cite{Neto:2017:STB:3106426.3106452} that found the toxic player to actually be a part of their online personality and that their very presence was in a way 'infecting' the other players and making them more toxic. \textit{Riot Games} developed a system which auto-detects profane language to root out the toxicity in their game \cite{Martens:2015:TDM:2984075.2984080}. Although, even this system cannot be fully relied on as self-depricating comments could be detected as toxic in this system \cite{Martens:2015:TDM:2984075.2984080}


The proposed CINDR(Complex, Indifferent, Necessity, Dualist, Responsive Feedback)\cite{7048084} idea for ethics in video games can be used to measure the depth of a game's moral decision making. A browser based game called Trolley Problem (https://www.pippinbarr.com/games/trolleyproblem/TrolleyProblem.html) emphasises the Dualist part of the CINDR idea; 2 decisions which are objectively neither good or evil. 


\section{Conclusion}
Video games have quickly become more relevant in the modern day world and with that surge in popularity, questions regarding the ethics of video games have come into question. The Trolley Problem \cite{7344559} poses the situation of 5 people being tied to a train line with the potential of switching the train line to one with only 1 person.
The easy option is to let the train carry on and ignore the whole situation, although in doing so you are consciously choosing to harm multiple people. This is a very primitive example of interactions with the online community in video games: either you exert your emotions on others and cause harm to multiple people or you keep them to yourself and harm only one person. 
The Trolley Problem has no right or wrong answer, which could be used as reasoning for why 'flaming' or 'griefing' also has no right or wrong answer. In addition to this leaving the punishments to a computer could result in mistakes as the computer punishes players who may have been aiming the insults at themselves, leaving the punishments to a person relies on the punishers morals and ethical background as to how severe the punishment would be.


\bibliography{references}
\bibliographystyle{plain}

\end{document}
This is never printed

