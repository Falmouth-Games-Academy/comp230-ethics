\title{Violence in games}

\author{James Pitt}

\date{\today}

\documentclass[12pt]{article}

\begin{document}
\maketitle
The difference in the DOOM controversy

\begin{abstract}
The video game industry has become the most prominent market in the world, generating more revenue last year than both the music and video industry. Developers for video games aim for audiences of all ages, some of which are of the more mature and violent nature.
There have been strong cases on both sides of the argument as to whether video games incite violent tendencies in people. In 2010 CA Anderson published an article stating that there would seem to be a link between people playing video games and an increase in violent behaviour.

In this article I will look at why the first DOOM in 1993 caused an uproar of controversy whereas the updated DOOM in 2016 caused little more than a small outburst. Also any correlation between density of video game sales and rate of crime and whether online multiplayer functionality helps reduce personal confrontation. Some scholars are arguing that video games such as Mortal Kombat and DOOM are a danger to public health as they increase aggression in younger people and encourage youth to replicate what they see on the screen.
\end{abstract}


\section{Influence on society}
Video games have quickly become more relevant in the modern day world than books as a survey suggests that in Australia, 2011, 92% of homes had access to a video gaming console
~\cite{Martin:2011:PGE:2071536.2071566}. Children are easily swayed by what they see and read as they have very little life experiences to compare them to and put them into contrast, therefore anything portrayed in the media that is accessible to children should make it very clear where to draw the line.
Some extreme points of view state that games like DOOM train teenagers and children to be able to fight at a military grade level, with examples of the Colorado shootings in the 1990s and the anti religious sentiment surrounding the killings dubbing the game as a “Mass Murder Simulator”~\cite{videogamescankill}.
The General Aggression Model believes that prolonged exposure to violent video games leads to an increase in general aggressive behaviour and higher expected hostility from people~\cite{Ashbarry:2016:BVE:2967934.2968111}~\cite{Fumhe:2015:VGP:2815782.2815790}. The critics of this statement say that it's down to the way the individual perceives it rather than the blanket statement that violent video games lead to a rise in aggression~\cite{5369086}. One of the reasons behind why they propose violent video games increase aggression is that players are faced with real life dilemmas with fictional or unreasonable solutions, an example of this is Grand Theft Auto V where the main character requires money so to solve the problem they rob a bank~\cite {Zendle:2015:HGF:2793107.2793113} ~\cite {Gotterbarn}.

\section{DOOM 1993-2000}
The original DOOM metaphorically shook the Earth when it came to breaking the mold in the video games industry.Some major events have been linked with the DOOM franchise and have been said to be the cause as to why the event happened in the first place. For example, the Columbine massacre has been said to be highly inspired by DOOM, among other titles. Families of the victims attempted to sue the game developers who they believe helped bring about the school shooting in the first place~\cite{bbc1} ~\cite{bbc2}. The case was thrown out of court by the judge as, at the time, video games were not subject to product liability laws.
However the controversy was not limited to just the original DOOM as the sequel to DOOM in 1995 also sparked controversy with the advertisements that the publishers used as seen in the 1995 Daily Mail article "Poor taste of Doom."~\cite{dailymail}. 

\section{DOOM 2016}
With the recent reboot to the DOOM series one would expect an equal, if not greater uproar from the increase in the graphical fidelity of the game. 
In a study by Zendle, it is shown that regardless of whether or not violent video games do encourage social violence and an increase in aggressive behaviour; the graphical fidelity of the game is irrelevant ~\cite{Zendle:2015:HGF:2793107.2793113}. While the effect on agressive behaviour that graphics has is still to be decided, it is clear that the graphical fidelity of a game is irrelevant.
An article written by the Andrew Griffin from The Independent newspaper in 2015 talked about the controversy which was created by this particular installment of the franchise. However, the criticism wasn't directed at the game so much as it was directed at the game; fans were cheering at the sight of gameplay where people were being sliced in half with a chainsaw. 
~\vspace{5mm}

\section{Conclusion}
Video games have quickly become more relevant in the modern day world than most forms of media, as such we can expect rapid development and change in the public's opinion on the ethics and morals of video game culture. Due to the original DOOM being released at a time where violent video games had only just started having an effect on the industry it is to be expected that it would pick up a lot of attention as it, in a way, pioneered the idea of violence in video games. When a group of fourth graders were asked about their gaming preferences over 50% of both boys and girls responded with a positive attitude towards increase violence~\cite{Chakraborty:2015:PPV:2717114.2692813}.


\bibliography{main}
\bibliographystyle{plain}

\end{document}
This is never printed

BBC Columbine
http://www.independent.co.uk/life-style/gadgets-and-tech/news/doom-launched-by-bethesda-at-e3-2015-swiftly-criticised-for-being-too-violent-10319985.html

Violence is just one of the many issues that video game developers have to be careful not to cross the line with when it comes to media reputation. Others include: Sexualisation, Gender, LGBT, Race and the portrayal of other Nations. The first sexually promiscuous game was called Softporn Adventure in 1980 ~\cite{softporn} and the overall reception of that game was fairly positive with the minor criticisms being that you could only seduce women and the parser "does not recognise the word woman". Interesting to note that only three years after the release of this game; Texas Chainsaw Massacre was released which showed a man holding a few blue pixels walking across the screen and when hitting an enemy the enemy would simply flip upside down. The official wiki for the game Texas Chainsaw Massacre states that due to the violent nature of the game quite a few stores refused to sell copies of the game ~\cite{tcm}. Comparably in 2008 a Japanese game called RapeLay sparked massive controversy in the UK and moved the British Parliament to enact 'Early Day Motion 818' which banned the sale of the game in the UK as they believed the simulation and gameplay of rape was detrimental to the players mental wellbeing. In the same year, the famous Call Of Duty World At War was released with the introduction of a new gamemode called Nazi Zombies which saw the violent massacre of hundreds of previously dead nazis however the game was met with very favourable reviews.
This could be due to public opinion changing over the years, or perhaps once the first game breaks new ground with controversial themes the media is less surprised the next time a game imitates the same themes

Inter-gaming public policy problems are an issue too, particularly over gender and the portrayal of female gamers in the community. Out of 74 women questioned, 18 percent said they felt a division between themselves and the male counterparts and/or had been verbally attacked by other gamers for just being female ~\cite{Shaer:2017:UGP:3025453.3025623}. One player reported being kicked out of a game after using a pink spray in Counter Strike: Global Offensive as the players discovered they were female ~\cite{Shaer:2017:UGP:3025453.3025623}. These statistics along with stories such as GamerGate ~\cite{Chatzakou:2017:MGT:3041021.3053890} highlight the clear division in gender and the drama created by this division, to the extent of death threats, in the online gaming world.